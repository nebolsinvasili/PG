\documentclass[oneside,14pt]{extarticle} % [односторонняя печать, 14pt шрифт]{расширенная статья}

%%% TOOLS
% Language setting
\usepackage[utf8]{inputenc} % Кодеровка текста UTF-8
\usepackage[T2A]{fontenc} % Шрифт русского текста
\usepackage[russian, english]{babel} % Подключаем языки

% Page setting
\usepackage{vmargin} % Подключаем колонтитулы
\setpapersize{A4} % Формат бумаги А4
\setmarginsrb{30mm}{20mm}{15mm}{20mm}{0pt}{0mm}{0mm}{8mm} % Размеры полей и колонтитулы

% Text setting
\usepackage{indentfirst} % Абзацный отступ (красная строка)
\sloppy % Включаем перенос слов

% Media setting
\usepackage{graphicx} % Для вставки картинок
\usepackage{amsmath} % Для вставки формул

% Code listing setting
\usepackage{listings} % Листинг кода
\lstset{language=Python,
	tabsize=2,
	breaklines,
	columns=fullflexible,
	flexiblecolumns,
	%numbers=left,
	numberstyle={\footnotesize},
	extendedchars=\true
}
%\lstset{, extendedchars=\true}

% Title setting
\usepackage{titlesec} % Переопределение заголовков
\titleformat{\section}{\filcenter\normalfont\Large\bfseries}{\thesection.}{0.6em}{}% Рамещение: {Заголовки}{По центру\Обычным шрифтом\Укрупненным\Жирным} {Форматирование: 1.}{Отступ от номера до текста: 0.6em}{}
\titleformat{\subsection}{\filright\normalfont\large\bfseries}{\thesubsection.}{0.4em}{}% Рамещение: {Заголовки}{По левому\Обычным шрифтом\Укрупненным(по меньше)\Жирным} {Форматирование: 1.}{Отступ от номера до текста: 0.4em}{}

\usepackage{datetime}
\newdateformat{yeardate}{\THEYEAR}


\begin{document} 

	\begin{titlepage} % Титульный лист
		
		\begin{center}
			{
				\fontsize{12}{12}\selectfont{
					\textbf{МИНИСТЕРСТВО НАУКИ И ВЫСШЕГО ОБРАЗОВАНИЯ
					\\РОССИЙСКОЙ ФЕДЕРАЦИИ}
					\\федеральное государственное автономное образовательное учреждение высшего образования
					\\«Национальный исследовательский технологический университет «МИСИС»
					
					
					\textbf{СТАРООСКОЛЬСКИЙ ТЕХНОЛОГИЧЕСКИЙ ИНСТИТУТ \\ИМ. А.А. УГАРОВА}
					
					
					(филиал) федерального государственного автономного образовательного учреждения
					\\высшего образования
					\\«Национальный исследовательский технологический университет «МИСИС»
					\\\textbf{(СТИ НИТУ «МИСИС»)}
					
					\medskip
					\textbf{ФАКУЛЬТЕТ АВТОМАТИЗАЦИИ И ИНФОРМАЦИОННЫХ ТЕХНОЛОГИЙ
					КАФЕДРА АВТОМАТИЗИРОВАННЫХ И ИНФОРМАЦИОННЫХ СИСТЕМ
					УПРАВЛЕНИЯ 
					\\ИМ. Ю.И. ЕРЕМЕНКО}
				
				\vspace{15mm}
				}
			}
			

			Домашняя работа №1
			\\по дисциплине: «Теория алгоритмов и структур данных»
			%\\на тему: «»
			
		\end{center}
		
		\vspace{30mm}
		
		\begin{flushleft}
			
			\fontsize{12}{12}\selectfont{
				Выполнил студент группы: \underline{АТ/МС-23Д, Небольсин Василий Дмитриевич\hspace{20mm}}
				\medskip
				\\\fontsize{10}{10}\selectfont{\hspace{80mm}группа, ФИО полностью\hspace{25mm}подпись}
			}
			\medskip
			
			\fontsize{12}{12}\selectfont{
				Проверил: \underline{ассистент кафедры АИСУ, Жуков Петр Игоревич\hspace{45mm}}
				\medskip
				\\\fontsize{10}{10}\selectfont{\hspace{60mm}Должность, звание, ФИО полностью\hspace{25mm}подпись}
			}
			
		\end{flushleft}
		\vfill
		\begin{center}
			
			\fontsize{12}{12}\selectfont{Старый Оскол, \yeardate\today}
			
		\end{center}
			
	\end{titlepage}
	\setcounter{page}{2}
	
	\begin{center}
		Задание
	\end{center}
	
	Реализовать алгоритм пузырьковой сортировки случайных значений длиной 10 и 100 тысяч. Оптимизировать метод до логарифмической вычислительной сложности.
	
	\begin{center}
		Решение
	\end{center}
	
	Пузырьковая сортировка, иногда называемая сортировкой по потоку, представляет собой простой алгоритм сортировки, который многократно перебирает входной список элемент за элементом, сравнивая текущий элемент с следующим за ним, меняя их значения при необходимости. Эти проходы по списку повторяются до тех пор, пока во время прохода не перестанут выполняться замены, что означает, что список стал полностью отсортированным. Алгоритм, представляющий собой сравнительную сортировку.
	
	Общая идея алгоритма состоит в следующем:
	
	Выбрать из массива элемент, называемый опорным. Это может быть любой из элементов массива. От выбора опорного элемента не зависит корректность алгоритма, но в отдельных случаях может сильно зависеть его эффективность (см. ниже).
	
	Сравнить все остальные элементы с опорным и переставить их в массиве так, чтобы разбить массив на три непрерывных отрезка, следующих друг за другом: «элементы меньшие опорного», «равные» и «большие»[2].
	
	Для отрезков «меньших» и «больших» значений выполнить рекурсивно ту же последовательность операций, если длина отрезка больше единицы.
	
	На практике массив обычно делят не на три, а на две части: например, «меньшие опорного» и «равные и большие»; такой подход в общем случае эффективнее, так как упрощает алгоритм разделения (листинг \ref{L_1}).
	
	\begin{lstlisting}[caption={Алгоритм пузырьковой сортировки}]
		def bubble_sort(arr):
			for i in range(len(arr)):
				for j in range(i+1, len(arr)):
					if arr[j] > arr[i]: # Comparing two list elements.
						arr[j], arr[i] = arr[i], arr[j] # Swap elements.
	\end{lstlisting}\label{L_1}
	
	Эта функция принимает на вход массив целых чисел и сортирует его с помощью пузырьковой сортировки.
		
	Функция сначала инициализирует переменную «i», чтобы отслеживать текущий индекс в массиве. Затем он проходит по всему массиву, используя два вложенных цикла for. На каждой итерации он сравнивает текущий элемент со следующим элементом и меняет их местами, если они находятся в неправильном порядке. Функция также обновляет индексы элементов по ходу работы.
		
	После завершения цикла функция возвращает отсортированный массив.
	
	
	Пузырьковая сортировка с рекурсией - это вариант пузырьковой сортировки, который использует рекурсию для повторного прохождения по списку до тех пор, пока он не будет полностью отсортирован.
	
	Рекурсивная пузырьковая сортировка начинается с обычного прохода по списку с помощью цикла for или while. Если в текущем проходе были произведены перестановки, то рекурсивно вызывается та же функция для неотсортированной части списка (листинг \ref{L_2}).
	
	\begin{lstlisting}[caption={Алгоритм рекурсивной пузырьковой сортировки}]
	def bubble_sort_recursive(arr, n):
		if n == 1:
			return
		for i in range(n-1):
			if arr[i] > arr[i+1]:
				arr[i], arr[i+1] = arr[i+1], arr[i]
		bubble_sort_recursive(arr, n-1)
	\end{lstlisting}\label{L_2}
	
	\begin{center}
		Результаты
	\end{center}
		
	Время работы пузырьковой сортировки со списками в 10 и 100 тысяч значений составил порядка 22 и 2328 секунд соответственно, что доказывает её неэффективность. 
	В тоже время алгоритм с рекурсией показал наилучший результат, справившись с темиже наборами за 13 и 1783 секунд.
	
\end{document}
