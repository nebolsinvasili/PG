\documentclass[oneside,14pt]{extarticle} % [односторонняя печать, 14pt шрифт]{расширенная статья}

%%% TOOLS
% Language setting
\usepackage[utf8]{inputenc} % Кодеровка текста UTF-8
\usepackage[T2A]{fontenc} % Шрифт русского текста
\usepackage[russian, english]{babel} % Подключаем языки

% Page setting
\usepackage{vmargin} % Подключаем колонтитулы
\setpapersize{A4} % Формат бумаги А4
\setmarginsrb{30mm}{20mm}{15mm}{20mm}{0pt}{0mm}{0mm}{8mm} % Размеры полей и колонтитулы

% Text setting
\usepackage{indentfirst} % Абзацный отступ (красная строка)
\sloppy % Включаем перенос слов

% Media setting
\usepackage{graphicx} % Для вставки картинок
\usepackage{amsmath} % Для вставки формул

% Code listing setting
\usepackage{listings} % Листинг кода
\lstset{language=Python,
	tabsize=2,
	breaklines,
	columns=fullflexible,
	flexiblecolumns,
	%numbers=left,
	numberstyle={\footnotesize},
	extendedchars=\true
}
%\lstset{, extendedchars=\true}

% Title setting
\usepackage{titlesec} % Переопределение заголовков
\titleformat{\section}{\filcenter\normalfont\Large\bfseries}{\thesection.}{0.6em}{}% Рамещение: {Заголовки}{По центру\Обычным шрифтом\Укрупненным\Жирным} {Форматирование: 1.}{Отступ от номера до текста: 0.6em}{}
\titleformat{\subsection}{\filright\normalfont\large\bfseries}{\thesubsection.}{0.4em}{}% Рамещение: {Заголовки}{По левому\Обычным шрифтом\Укрупненным(по меньше)\Жирным} {Форматирование: 1.}{Отступ от номера до текста: 0.4em}{}

\usepackage{datetime}
\newdateformat{yeardate}{\THEYEAR}


\begin{document} 

	\begin{titlepage} % Титульный лист
		
		\begin{center}
			{
				\fontsize{12}{12}\selectfont{
					\textbf{МИНИСТЕРСТВО НАУКИ И ВЫСШЕГО ОБРАЗОВАНИЯ
					\\РОССИЙСКОЙ ФЕДЕРАЦИИ}
					\\федеральное государственное автономное образовательное учреждение высшего образования
					\\«Национальный исследовательский технологический университет «МИСИС»
					
					
					\textbf{СТАРООСКОЛЬСКИЙ ТЕХНОЛОГИЧЕСКИЙ ИНСТИТУТ \\ИМ. А.А. УГАРОВА}
					
					
					(филиал) федерального государственного автономного образовательного учреждения
					\\высшего образования
					\\«Национальный исследовательский технологический университет «МИСИС»
					\\\textbf{(СТИ НИТУ «МИСИС»)}
					
					\medskip
					\textbf{ФАКУЛЬТЕТ АВТОМАТИЗАЦИИ И ИНФОРМАЦИОННЫХ ТЕХНОЛОГИЙ
					КАФЕДРА АВТОМАТИЗИРОВАННЫХ И ИНФОРМАЦИОННЫХ СИСТЕМ
					УПРАВЛЕНИЯ 
					\\ИМ. Ю.И. ЕРЕМЕНКО}
				
				\vspace{15mm}
				}
			}
			

			Домашняя работа №2
			\\по дисциплине: «Теория алгоритмов и структур данных»
			%\\на тему: «»
			
		\end{center}
		
		\vspace{30mm}
		
		\begin{flushleft}
			
			\fontsize{12}{12}\selectfont{
				Выполнил студент группы: \underline{АТ/МС-23Д, Небольсин Василий Дмитриевич\hspace{20mm}}
				\medskip
				\\\fontsize{10}{10}\selectfont{\hspace{80mm}группа, ФИО полностью\hspace{25mm}подпись}
			}
			\medskip
			
			\fontsize{12}{12}\selectfont{
				Проверил: \underline{ассистент кафедры АИСУ, Жуков Петр Игоревич\hspace{45mm}}
				\medskip
				\\\fontsize{10}{10}\selectfont{\hspace{60mm}Должность, звание, ФИО полностью\hspace{25mm}подпись}
			}
			
		\end{flushleft}
		\vfill
		\begin{center}
			
			\fontsize{12}{12}\selectfont{Старый Оскол, \yeardate\today}
			
		\end{center}
			
	\end{titlepage}
	\setcounter{page}{2}
	
	\begin{center}
		Задание
	\end{center}
	
	Реализовать Binary Search Tree со значениями от 1 до 20 и алгоритм поиска в глубину.
	
	\begin{center}
		Решение
	\end{center}
	
	Реализация дерева поиска идентична связномы списку с единственным отличием: узел списка указывает на один другой узел, а дерево на большее их число. В последних каждый родительский узел может иметь несколько узлов-потомков. Если у каждого узла максимум два узла-потомка (левый и правый), такое дерево называется двоичным или бинарным.
	
	Класс узла будет содержать в себе три атрибута: значение узла, правый и левый указатель.
	
	\begin{lstlisting}[caption={Реализация узла дерева поиска}]
		class Node:
			def __init__(self, value):
				self.value = value
				self.left = None
				self.right = None
	\end{lstlisting}
	
	Бинарное дерево поиска следует определенному порядку расположения элементов. В дереве двоичного поиска значение левого узла должно быть меньше родительского узла, а значение правого узла больше. Это правило применяется рекурсивно к левому и правому поддеревьям корня.
	
	\begin{lstlisting}[caption={Вставка элемента в бинарное дерево}]
		def insert(root, value):
			if root is None:
				return Node(value)
			elif value < root.value:
				root.left = insert(root.left, value)
			else:
				root.right = insert(root.right, value)
			return root
	\end{lstlisting}
	
	Рассмотрим алгоритм создание бинарного дерева поиска.
	
	\begin{lstlisting}[caption={Алгоритм создания BST}]
		def create_binary_tree(values):
		root = Node(values[0])
		for value in values[1:]:
		insert(root, Node(value))
		return root
	\end{lstlisting}
	
	Сначала нужно вставить корень дерева.
	Затем прочитать следующий элемент. Если он меньше корневого узла, вставить его как корень левого поддерева и перейдите к следующему элементу, в противном случае как корень правого поддерева.
	
	На этом создание бинарного дерева поиска завершено. Теперь перейдем к алгоритмам поиска, которые можно выполнять с BST.
	
	Самые простые в реализации обхода дерева — прямой (Pre-Order), обратный (Post-Order) и центрированный (In-Order), также поиск в ширину и поиск в глубину.
	
	При обходе в глубину (Depth First Search, DFS) алгоритм сначала опускается к низу дерева, а потом идет в сторону, а при обходе в ширину (Breadth First Search, BFS) — наоборот, начинает с корня и обходит узлы-потомки, потом спускается к потомкам потомков, и так далее.
	
	\begin{lstlisting}[caption={Алгоритм поиска в глубину}]
		def depth_first_search(node, value):
			if node is None or node.value == value:
				return node
			elif value < node.value:
				return depth_first_search(node.left, value)
			else:
			return depth_first_search(node.right, value)
	\end{lstlisting}
	
	
	\begin{center}
		Результаты
	\end{center}
		
	В отличии от массивов и связанных списков двоичное дерево поиска имеет ряд преимуществ.

	Операции с деревом работают быстрее. При реализации списком все функции требуют $O(n)$ действий, где n — размер структуры. Операции с деревом же работают за $O(h)$, где h — максимальная глубина дерева (глубина — расстояние от корня до вершины). 
	
	В оптимальном случае, когда глубина всех листьев одинакова, в дереве будет $n=2^h$ вершин. Значит, сложность операций в деревьях, близких к оптимуму будет $O(log(n))$. 
	
	К сожалению, в худшем случае дерево может выродится и сложность операций будет как у списка, например если вставлять числа $1..n$ по порядку.
	
	Однако существуют способы реализовать дерево так, чтобы оптимальная глубина дерева сохранялась при любой последовательности операций. Такие деревья называют сбалансированными. К ним например относятся красно-черные деревья, AVL-деревья, splay-деревья
	
\end{document}
